% ============================================================================
%  MCKL/manual/tex/perf_random_distribution.tex
% ----------------------------------------------------------------------------
%  MCKL: Monte Carlo Kernel Library
% ----------------------------------------------------------------------------
%  Copyright (c) 2013-2016, Yan Zhou
%  All rights reserved.
%
%  Redistribution and use in source and binary forms, with or without
%  modification, are permitted provided that the following conditions are met:
%
%    Redistributions of source code must retain the above copyright notice,
%    this list of conditions and the following disclaimer.
%
%    Redistributions in binary form must reproduce the above copyright notice,
%    this list of conditions and the following disclaimer in the documentation
%    and/or other materials provided with the distribution.
%
%  THIS SOFTWARE IS PROVIDED BY THE COPYRIGHT HOLDERS AND CONTRIBUTORS "AS IS"
%  AND ANY EXPRESS OR IMPLIED WARRANTIES, INCLUDING, BUT NOT LIMITED TO, THE
%  IMPLIED WARRANTIES OF MERCHANTABILITY AND FITNESS FOR A PARTICULAR PURPOSE
%  ARE DISCLAIMED. IN NO EVENT SHALL THE COPYRIGHT HOLDER OR CONTRIBUTORS BE
%  LIABLE FOR ANY DIRECT, INDIRECT, INCIDENTAL, SPECIAL, EXEMPLARY, OR
%  CONSEQUENTIAL DAMAGES (INCLUDING, BUT NOT LIMITED TO, PROCUREMENT OF
%  SUBSTITUTE GOODS OR SERVICES; LOSS OF USE, DATA, OR PROFITS; OR BUSINESS
%  INTERRUPTION) HOWEVER CAUSED AND ON ANY THEORY OF LIABILITY, WHETHER IN
%  CONTRACT, STRICT LIABILITY, OR TORT (INCLUDING NEGLIGENCE OR OTHERWISE)
%  ARISING IN ANY WAY OUT OF THE USE OF THIS SOFTWARE, EVEN IF ADVISED OF THE
%  POSSIBILITY OF SUCH DAMAGE.
% ============================================================================

\chapter{Performance of distribution generators}
\label{chap:Performance of distribution generators}

The performance of distributions in Section~\ref{sec:Distributions} are
measured in single core cycles per element (cpE). The processor used is Intel
Core i7-4960\textsc{hq}. The operating system is Mac OS X (version~10.11.4).
The compiler is \llvm{} clang (version Apple~7.3.0). The \mkl library is used
for vectorized mathematical functions.

Four usage cases of the distributions are considered. First, if the
distribution is available in the standard library, we measure the following
case,
\begin{Verbatim}
  std::normal_distribution<double> rnorm_std(0, 1);
  for (std::size_t i = 0; i = n; ++i)
      r[i] = rand(rng, rnorm_std);
\end{Verbatim}
Second, we measure the performance of the library's implementation,
\begin{Verbatim}
  NormalDistribution<double> rnorm_mckl(0, 1);
  for (std::size_t i = 0; i = n; ++i)
      r[i] = rand(rng, rnorm_mckl);
\end{Verbatim}
The third is the vectorized performance,
\begin{Verbatim}
  rand(rng, rnorm_mckl, n, r.data());
\end{Verbatim}
For all the three above, the \rng is \verb|ARSx8| (Section~\ref{sub:AES-NI
  instructions based RNG}). The last is when \rng is \verb|MKL_SFMT19937|
(Section~\ref{sec:MKL RNG}),
\begin{Verbatim}
  MKL_SFMT19937 rng_mkl;
  rand(rng_mkl, rnorm_mckl, n, r.data());
\end{Verbatim}
In all cases, we repeat the simulations 100 times, each time with the number of
elements $n$ chosen randomly between 5,000 and 10,000. The total number of
cycles of the 100 simulations are recorded, and then divided by the total
number of elements generated. This gives the performance measurement in cpE.
This experiment is repeated ten times, and the best results are shown. The four
cases are labeled ``\std'', ``\mckl'', ``Batch'' and ``\mkl'', respectively, in
tables below.

\begin{table}
  \input{tab/random_distribution_inverse}%
  \caption{Performance of distributions using the inverse method}
  \label{tab:Performance of distributions using the inverse method}
\end{table}

\begin{table}
  \input{tab/random_distribution_beta}%
  \caption{Performance of Beta distribution}
  \label{tab:Performance of Beta distribution}
\end{table}

\begin{table}
  \input{tab/random_distribution_chisquared}%
  \caption{Performance of $\chi^2$ distribution}
  \label{tab:Performance of chi-squared distribution}
\end{table}

\begin{table}
  \input{tab/random_distribution_gamma}%
  \caption{Performance of Gamma distribution}
  \label{tab:Performance of Gamma distribution}
\end{table}

\begin{table}
  \input{tab/random_distribution_fisherf}%
  \caption{Performance of Fisher's $F$-distribution}
  \label{tab:Performance of Fisher's F-distribution}
\end{table}

\begin{table}
  \input{tab/random_distribution_normal}%
  \caption{Performance of Normal and related distributions}
  \label{tab:Performance of Normal and related distributions}
\end{table}

\begin{table}
  \input{tab/random_distribution_studentt}%
  \caption{Performance of Student's $t$-distribution}
  \label{tab:Performance of Student's t-distribution}
\end{table}

\begin{table}
  \input{tab/random_distribution_int}%
  \caption{Performance of discrete distributions}
  \label{tab:Performance of discrete distributions}
\end{table}
