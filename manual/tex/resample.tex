% ============================================================================
%  MCKL/manual/tex/resample.tex
% ----------------------------------------------------------------------------
%  MCKL: Monte Carlo Kernel Library
% ----------------------------------------------------------------------------
%  Copyright (c) 2013-2016, Yan Zhou
%  All rights reserved.
%
%  Redistribution and use in source and binary forms, with or without
%  modification, are permitted provided that the following conditions are met:
%
%    Redistributions of source code must retain the above copyright notice,
%    this list of conditions and the following disclaimer.
%
%    Redistributions in binary form must reproduce the above copyright notice,
%    this list of conditions and the following disclaimer in the documentation
%    and/or other materials provided with the distribution.
%
%  THIS SOFTWARE IS PROVIDED BY THE COPYRIGHT HOLDERS AND CONTRIBUTORS "AS IS"
%  AND ANY EXPRESS OR IMPLIED WARRANTIES, INCLUDING, BUT NOT LIMITED TO, THE
%  IMPLIED WARRANTIES OF MERCHANTABILITY AND FITNESS FOR A PARTICULAR PURPOSE
%  ARE DISCLAIMED. IN NO EVENT SHALL THE COPYRIGHT HOLDER OR CONTRIBUTORS BE
%  LIABLE FOR ANY DIRECT, INDIRECT, INCIDENTAL, SPECIAL, EXEMPLARY, OR
%  CONSEQUENTIAL DAMAGES (INCLUDING, BUT NOT LIMITED TO, PROCUREMENT OF
%  SUBSTITUTE GOODS OR SERVICES; LOSS OF USE, DATA, OR PROFITS; OR BUSINESS
%  INTERRUPTION) HOWEVER CAUSED AND ON ANY THEORY OF LIABILITY, WHETHER IN
%  CONTRACT, STRICT LIABILITY, OR TORT (INCLUDING NEGLIGENCE OR OTHERWISE)
%  ARISING IN ANY WAY OUT OF THE USE OF THIS SOFTWARE, EVEN IF ADVISED OF THE
%  POSSIBILITY OF SUCH DAMAGE.
% ============================================================================

\chapter{Resampling}
\label{chap:Resampling}

Given a particle system $S_{1:N}$, $S_i = (X_i,W_i)$, a resampling algorithm
generate a new system $\hat{S}_{1:M}$ such that,
\begin{equation*}
  \Exp\Square[Big]{\sum_{i=1}^M{\hat{W}_i\varphi(\hat{X}_i)}} =
  \Exp\Square[Big]{\sum_{i=1}^N{W_i\varphi(X_i)}}
\end{equation*}
for test function $\varphi$. Regardless of other statistical properties, in
practice, such an algorithm can be decomposed into three steps,
\begin{enumerate}
  \item Generate an $N$-vector of replication numbers $r_{1:N}$, such that
    $\sum_{i=1}^N r_i = M$, and $0 \le r_i \le M$ for $i=1,\dots,N$.
  \item Generate an $M$-vector of indices $a_{1:M}$ such that $\sum_{j=1}^M
    \bbI_{\{i\}}(a_j) = r_i$, and $1 \le a_i \le N$ for $i = 1,\dots,M$.
  \item Set $\hat{X}_i = X_{a_i}$ for $i = 1,\dots,M$.
\end{enumerate}
Given the results of the first step, the second step can be implemented with a
deterministic algorithm. And the final results will be determined up to
re-ordering. Therefore, it is the first step that determines the statistical
properties of the new particle system.

\section{Using resampling with sampler}
\label{sec:Using resampling with sampler}

Recall Section~\ref{sec:Sampler}, any object that is convertible to,
\begin{Verbatim}
  using eval_type =
      std::function<void(size_t, Particle<T> &)>;
\end{Verbatim}
can be added as a resampling evaluation object to a \verb|Sampler| object. The
library defines the following class template,
\begin{Verbatim}
  template <typename T>
  class ResampleEval;
\end{Verbatim}
that is compatible with the type above. It has a single constructor,
\begin{Verbatim}
  explicit ResampleEval(const eval_type &eval);
\end{Verbatim}
where \verb|eval_type| (not to be confused with the type of the same name in
\verb|Sampler|) is defined as the following,
\begin{Verbatim}
  using eval_type = std::function<void(
          size_t,
          size_t,
          typename Particle<T>::rng_type &,
          const double *,
          typename Particle<T>::size_type *)>;
\end{Verbatim}
An evaluation object that is convertible to the above will be used to generate
the $N$-vector of replication numbers $r_{1:N}$. When called, it will be passed
the following arguments,
\begin{Verbatim}
  eval(N, M, rng, w, r);
\end{Verbatim}
where $N$ is the original sample size, $M$ is the new sample size, \verb|rng|
is an \rng engine, \verb|w| is a pointer to the $N$-vector of normalized
weights, and output parameter \verb|r| points to the $N$-vector of replication
numbers. One can define function templates to avoid declaring these parameter
types explicitly. An object of the class \verb|ResampleEval| is convertible to
\verb|eval_type| of \verb|Sampler|, and can be added to a sampler as a
resampling evaluation object. Its operator will call the object passed to its
constructor to generate the $N$-vector of replication numbers $r_{1:N}$. And it
will generate the the $M$-vector of indices $a_{1:M}$. And last, it will used
the \verb|select| method of type \verb|T| to duplicate states. The vector of
indices generated by this operator has the following property, in addition to
those stated earlier at the beginning of this chapter,
\begin{equation*}
  a_i = i \quad \text{if} \quad  r_i > 0 \quad
  \text{for } i = 1,\dots,\min\{N, M\}
\end{equation*}
In fact, the \verb|select| method of \verb|StateMatrix| in
Section~\ref{sec:State} makes the assumptions of this property about its input
indices.

\section{Algorithm}
\label{sec:Algorithm}

The library implements all algorithms discussed in \cite{Douc:2005wa} and two
extensions to the those algorithms. Samplers can be constructed with builtin
algorithms as seen in Section~\ref{sec:Sampler}. Builtin algorithms are
implemented in the following class template,
\begin{Verbatim}
  template <typename U01SeqType, bool Residual>
  class ResampleAlgorithm;
\end{Verbatim}
We will explain the template parameters later. This class has the following
interface,
\begin{Verbatim}
  template <
      typename RNGType,
      typename InputIter,
      typename OutputIter>
  void eval(
      size_t N,
      size_t M,
      RNGType &rng,
      InputIter w,
      OutputIter r) const
\end{Verbatim}
Its parameters are as those described earlier for \verb|eval_type| of
\verb|ResampleEval|. The algorithm that it implements depends on the template
parameters of the class. The type \verb|U01SeqType| shall be a class type with
a default constructor and a call operator. It shall be able to be used as the
following,
\begin{Verbatim}
  typename std::iterator_traits<IntputIter>::value_type *u;
  // allocate space for u
  U01SeqType u01seq;
  u01seq(rng, R, u);
\end{Verbatim}
After the call, it shall generate a sequence $0 \le U_1 \le \dots\le U_R < 1$.
The library provides three implementations, which will be discussed in the next
section. The algorithm proceeds as the following to generate $r_{1:N}$,
\begin{algorithmic}
  \REQUIRE $\sum_{i=1}^N W_i = 1$
  \IF{\texttt{Residual} is false}
  \STATE $r_i \leftarrow 0$ for $i = 1,\dots,N$
  \STATE $R \leftarrow M$
  \ELSE
  \STATE $r_i \leftarrow \Floor{MW_i}$ for $i = 1,\dots,N$
  \STATE $R \leftarrow M - \sum_{i=1}^N r_i$
  \STATE $W_i \leftarrow MW_i - r_i$ for $i = 1,\dots,N$
  \STATE $W_i \leftarrow W_i / \sum_{i=1}^NW_i$
  \ENDIF
  \STATE Generate $U_{1:R}$ using \verb|U01SeqType|
  \STATE $V_0 \leftarrow 0$, $V_i \leftarrow V_{i - 1} + W_i$ for $i =
  1,\dots,N$.
  \STATE $r_i \leftarrow r_i + \sum_{j=1}^R\bbI_{[V_{i-1},V_i)}(U_j)$
\end{algorithmic}
Builtin schemes differ in their choices of the template parameters
\verb|U01SeqType| and \verb|Residual|. There are three implementations in the
library of the uniform sequence,
\begin{Verbatim}
  class U01SequenceSorted;
\end{Verbatim}
generates the sequence $U_{1:R}$ such that it has the same distribution as a
sorted sequence of i.i.d.\ standard uniform random variables $V_{1:R}$.
\begin{Verbatim}
  class U01SequenceStratified;
\end{Verbatim}
generates the sequence $U_{1:R}$ such that $U_i = (i - 1)\delta + V_i\delta$
where $V_{1:R}$ are i.i.d.\ standard uniform random variables and $\delta = 1 /
R$.
\begin{Verbatim}
  class U01SequenceSystematic;
\end{Verbatim}
generates the sequence $U_{1:R}$ such that $U_i = (i - 1)\delta + V\delta$
where $V$ is a standard uniform random variable and $\delta = 1 / R$.

All the builtin algorithms are listed in Table~\ref{tab:Resampling schemes}.
Convenient type aliases are also defined,
\begin{Verbatim}
  using ResampleMultinomial =
      ResampleAlgorithm<U01SequenceSorted, false>;

  using ResampleStratified =
      ResampleAlgorithm<U01SequenceStratified, false>;

  using ResampleSystematic =
      ResampleAlgorithm<U01SequenceSystematic, false>;

  using ResampleResidual =
      ResampleAlgorithm<U01SequenceSorted, true>;

  using ResampleResidualStratified =
      ResampleAlgorithm<U01SequenceStratified, true>;

  using ResampleResidualSystematic =
      ResampleAlgorithm<U01SequenceSystematic, true>;
\end{Verbatim}

\begin{table}
  \begin{tabularx}{\textwidth}{lLl}
    \toprule
    \verb|ResampleScheme| & \verb|U01SeqType| & \verb|Residual| \\
    \midrule
    \verb|Multinomial|        & \verb|U01SequenceSorted|     & \verb|false| \\
    \verb|Stratified|         & \verb|U01SequenceStratified| & \verb|false| \\
    \verb|Systematic|         & \verb|U01SequenceSystematic| & \verb|false| \\
    \verb|Residual|           & \verb|U01SequenceSorted|     & \verb|true|  \\
    \verb|ResidualStratified| & \verb|U01SequenceStratified| & \verb|true|  \\
    \verb|ResidualSystematic| & \verb|U01SequenceSystematic| & \verb|true|  \\
    \bottomrule
  \end{tabularx}
  \caption{Resampling schemes}
  \label{tab:Resampling schemes}
\end{table}
