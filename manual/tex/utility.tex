\chapter{Utilities}
\label{chap:Utilities}

The library provides some utilities for writing Monte Carlo simulation
programs. For some of them, such as command line option processing, there are
more advanced, dedicated libraries out there. The library only provides some
basic functionality that is sufficient for simple cases.

\section{Aligned memory allocation}
\label{sec:Aligned memory allocation}

The standard library class \verb|std::allocator| is used by containers to
allocate memory. It works fine in most cases. However, sometime it is desirable
to allocate memory aligned by a certain boundary. The library provides the
class template,
\begin{Verbatim}
  template <typename T, std::size_t Alignment = AlignmentTrait<T>::value,
      typename Memory = AlignedMemory>
  class Allocator;
\end{Verbatim}
which conforms to the \verb|std::allocator| interface. The address of the
pointer returned by the \verb|allocate| method will be a multiple of
\verb|Alignment|. The value of alignment has to be positive, larger than
\verb|sizeof(void *)|, and a power of two. Violating any of these conditions
will result in compile-time error. The last template parameter \verb|Memory|
shall have two static methods,
\begin{Verbatim}
  static void *aligned_malloc(std::size_t n, std::size_t alignment);
  static void aligned_free(void *ptr);
\end{Verbatim}
The method \verb|aligned_malloc| shall behave similar to \verb|std::malloc|
with the additional alignment requirement. It shall return a null pointer if it
fails to allocate memory. In any other case, including zero input size, it
shall return a reachable non-null pointer. The method \verb|aligned_free| shall
behave similar to \verb|std::free|. It shall be able to handle a null pointer
as its input. The library provides three implementations, discussed below. The
default argument \verb|AlignedMemory| is an alias to one of them by default. It
can be changed by defining the macro \verb|MCKL_ALIGNED_MEMORY_TYPE|.

\paragraph{\texttt{AlignedMemoryTBB}} This class uses
\verb|scalable_aligned_malloc| and \verb|scalable_aligned_free| from the \tbb
library. This is the default method if \verb|MCKL_USE_TBB_MALLOC| is true.

\paragraph{\texttt{AlignedMemorySYS}} This class uses \verb|posix_memalign| and
\verb|free| on \posix platforms. It uses \verb|_aligned_malloc| and
\verb|_aligned_free| if the compiler is \msvc. Otherwise, this class is not
defined. If this class is define, it is the default method if
\verb|MCKL_USE_TBB_MALLOC| is false.

\paragraph{\texttt{AlignedMemorySTD}} This class uses only \verb|std::malloc|
and \verb|std::free|. It is the last resort method the library will use.

The default alignment depends on the type \verb|T|. If it is a scalar type
(\verb|std::is_scalar<T>|), then the alignment is \verb|MCKL_ALIGNMENT|, whose
default is 32. This alignment is sufficient for modern \simd operations, such
as \avx. For other types, the alignment is the maximum of \verb|alignof(T)| and
\verb|MCKL_ALIGNMENT_MIN|, whose default is 16. Two container types are defined
for convenience,
\begin{Verbatim}
  template <typename T>
  using Vector = std::vector<T, Allocator<T>>;

  template <typename T, std::size_t N,
      std::size_t Alignment = AlignmentTrait<T>::value>
  class alignas(Alignment) Array;
\end{Verbatim}
The first can be used as a drop-in replacement of \verb|std::vector<T>| since
it is merely a type alias with a different default allocator. The second can is
derived from \verb|std::array<T, N>| and thus can be a drop-in replacement in
most situations.

\section{Sample covariance}
\label{sec:Sample covariance}

The library provides a class template to estimate sample covariance,
\begin{Verbatim}
  template <typename RealType>
  class Covariance;
\end{Verbatim}
At the time of writting, only \verb|float| and \verb|double| are supported.
The class has the following operator as its interface,
\begin{Verbatim}
  void operator()(MatrixLayout layout, std::size_t n, std::size_t p,
      const RealType *x, const RealType *w, RealType *mean,
      RealType *cov, MatrixLayout cov_layout = RowMajor,
      bool cov_upper = false, bool cov_packed = false)
\end{Verbatim}
It computes the sample covariance matrix $\Sigma$,
\begin{align*}
  \Sigma_{i,j} &= \frac{\sum_{i=1}^N w_i}
  {(\sum_{i=1}^N w_i)^2 - \sum_{i=1}^n w_i^2}
  \sum_{k=1}^N w_k (x_{k,i} - \bar{x}_i)(x_{k,j} - \bar{x}_j) \\
  \bar{x}_i &= \frac{1}{\sum_{i=1}^N w_i}\sum_{k=1}^N w_k x_{k,i}
\end{align*}
where $x$ is the $n$ by $p$ matrix of samples, and $w$ is the $n$-vector of
weights. Below we given detailed description of each of the parameters,
\begin{description}
  \item[\texttt{layout}] The storage layout of sample matrix $x$. It is assumed
    to be an $n$ by $p$ matrix.
  \item[\texttt{n}] The number of samples.
  \item[\texttt{p}] The dimension of each sample.
  \item[\texttt{x}] The sample matrix. If it is a null pointer, then no
    computation is carried out.
  \item[\texttt{w}] The weight vector. If it is a null pointer, then all
    samples are assigned weight $1$.
  \item[\texttt{mean}] Output storage of the mean. If it is a null pointer,
    then it is ignored.
  \item[\texttt{cov}] Output storage of the covariance matrix. If it is a null
    pointer, then it is ignored.
  \item[\texttt{cov\_layout}] The storage layout of the covariance matrix.
  \item[\texttt{cov\_upper}] If \verb|true|, then the upper triangular of the
    covariance matrix is packed, otherwise the lower triangular is packed.
    Ignored if \verb|cov_pack| is \verb|false|.
  \item[\texttt{cov\_packed}] If \verb|true|, then the covariance matrix is
    packed.
\end{description}
The last three parameters specify the storage scheme of the covariance matrix.
See any reference of \blas or \lapack for explanation of the scheme. Below is
an example of the class in use,
\begin{Verbatim}
  using T = StateMatrix<RowMajor, Dynamic, double>;
  Sampler<T> sampler(n, p);
  // Configure and iterate the sampler
  double mean[p];
  double cov[p * (p + 1) / 2];
  Covariance eval;
  auto x = sampler.particle().value().data();
  auto w = sampler.particle().weight().data();
  eval(RowMajor, n, p, x, w, mean, cov, RowMajor, false, true);
\end{Verbatim}
One can later compute the Cholesky decomposition using \lapack or other linear
algebra libraries. Below is an example of using the covariance matrix to
generate multivariate Normal proposals,
\begin{Verbatim}
  double chol[p * (p + 1) / 2];
  double y[p];
  LAPACKE_dpptrf(LAPACK_ROW_MAJOR, 'L', p, chol);
  NormalMVDistribution<double, p> normal_mv(mean, chol);
  normal_mv(rng, y);
\end{Verbatim}

\section{Store objects in \protect\hdf5 format}
\label{sec:Store objects in HDF5 format}

If the \hdf library is available (\verb|MCKL_HAS_HDF5|), it is possible to
store \verb|Sampler<T>| objects, etc., in the \hdf format. For example,
\begin{Verbatim}
  hdf5store(sampler, "pf.h5", "sampler", false);
\end{Verbatim}
creates an \hdf file named \verb|pf.h5| with the sampler stored as a list in
the group \verb|sampler|. If the last argument is \verb|true|, the data is
inserted to an existing file. Otherwise a new file is created. In R it can be
processed as the following,
\begin{Verbatim}
  library(rhdf5)
  pf <- as.data.frame(h5read("pf.h5", "sampler"))
\end{Verbatim}
This creates a \verb|data.frame| similar to that shown in
section~\ref{sub:Implementations}. The \verb|hdf5store| function is overloaded
for \verb|StateMatrix|, \verb|Sampler<T>| and \verb|Monitor<T>|. It is also
overloaded for \verb|Particle<T>| if an overload for \verb|T| is defined. The
all follow the format as above. In addition, the following creates a new empty
\hdf file,
\begin{Verbatim}
  hdf5store(filename);
\end{Verbatim}
while the following create a new group named \verb|dataname|,
\begin{Verbatim}
  hdf5store(filename, dataname, append);
\end{Verbatim}

\section{Stop watch}
\label{sec:Stop watch}

Performance can only be improved after it is first properly benchmarked. There
are advanced profiling programs for this purpose. However, sometime simple
timing facilities are enough. The library provides a simple class
\verb|StopWatch| for this purpose. As its name suggests, it works much like a
physical stop watch. Here is a simple example
\begin{Verbatim}
  StopWatch watch;
  for (std::size_t i = 0; i != n; ++i) {
      // Some computation
      watch.start();
      // Computation to be benchmarked;
      watch.stop();
      // Some other computation
  }
  double t = watch.seconds(); // The time in seconds
\end{Verbatim}
The above example demonstrate that timing can be accumulated between loop
iterations, function calls, etc. It shall be noted that, the timing is only
accurate if the computation between \verb|start| and \verb|stop| is
non-trivial.
