% ============================================================================
%  MCKL/manual/tex/utility.tex
% ----------------------------------------------------------------------------
%  MCKL: Monte Carlo Kernel Library
% ----------------------------------------------------------------------------
%  Copyright (c) 2013-2016, Yan Zhou
%  All rights reserved.
%
%  Redistribution and use in source and binary forms, with or without
%  modification, are permitted provided that the following conditions are met:
%
%    Redistributions of source code must retain the above copyright notice,
%    this list of conditions and the following disclaimer.
%
%    Redistributions in binary form must reproduce the above copyright notice,
%    this list of conditions and the following disclaimer in the documentation
%    and/or other materials provided with the distribution.
%
%  THIS SOFTWARE IS PROVIDED BY THE COPYRIGHT HOLDERS AND CONTRIBUTORS "AS IS"
%  AND ANY EXPRESS OR IMPLIED WARRANTIES, INCLUDING, BUT NOT LIMITED TO, THE
%  IMPLIED WARRANTIES OF MERCHANTABILITY AND FITNESS FOR A PARTICULAR PURPOSE
%  ARE DISCLAIMED. IN NO EVENT SHALL THE COPYRIGHT HOLDER OR CONTRIBUTORS BE
%  LIABLE FOR ANY DIRECT, INDIRECT, INCIDENTAL, SPECIAL, EXEMPLARY, OR
%  CONSEQUENTIAL DAMAGES (INCLUDING, BUT NOT LIMITED TO, PROCUREMENT OF
%  SUBSTITUTE GOODS OR SERVICES; LOSS OF USE, DATA, OR PROFITS; OR BUSINESS
%  INTERRUPTION) HOWEVER CAUSED AND ON ANY THEORY OF LIABILITY, WHETHER IN
%  CONTRACT, STRICT LIABILITY, OR TORT (INCLUDING NEGLIGENCE OR OTHERWISE)
%  ARISING IN ANY WAY OUT OF THE USE OF THIS SOFTWARE, EVEN IF ADVISED OF THE
%  POSSIBILITY OF SUCH DAMAGE.
% ============================================================================

\chapter{Utilities}
\label{chap:Utilities}

The library provides some utilities for writing Monte Carlo simulation
programs.

\section{Aligned memory allocation}
\label{sec:Aligned memory allocation}

The standard library class \verb|std::allocator| is used by containers to
allocate memory. It works fine in most cases. However, sometime it is desirable
to allocate memory aligned by a certain boundary. The library provides the
class template,
\begin{Verbatim}
  template <
      typename T,
      size_t Alignment = AlignmentTrait<T>::value,
      typename Memory = AlignedMemory>
  class Allocator;
\end{Verbatim}
which conforms to the \verb|std::allocator| interface. The address of the
pointer returned by the \verb|allocate| method will be a multiple of
\verb|Alignment|. The value of alignment has to be positive, larger than
\verb|sizeof(void *)|, and a power of two. Violating any of these conditions
will result in compile-time error. The last template parameter \verb|Memory|
shall have two static methods,
\begin{Verbatim}
  static void *aligned_malloc(size_t n, size_t alignment);
  static void aligned_free(void *ptr);
\end{Verbatim}
The method \verb|aligned_malloc| shall behave similar to \verb|std::malloc|
with the additional alignment requirement. It shall return a null pointer if it
fails to allocate memory. In any other case, including zero input size, it
shall return a reachable non-null pointer. The method \verb|aligned_free| shall
behave similar to \verb|std::free|. It shall be able to handle a null pointer
as its input. The library provides three implementations, discussed below. The
default argument \verb|AlignedMemory| is an alias to one of them by default. It
can be changed by defining the macro \verb|MCKL_ALIGNED_MEMORY_TYPE|.

\paragraph{\texttt{AlignedMemoryTBB}} This class uses the functions
\verb|scalable_aligned_malloc| and \verb|scalable_aligned_free| from the \tbb
library.
This is the default method if \verb|MCKL_USE_TBB_MALLOC| is true.

\paragraph{\texttt{AlignedMemorySYS}} On \posix platforms, this class uses
\verb|posix_memalign| and \verb|free|. If the compiler is \msvc, it uses
\verb|_aligned_malloc| and \verb|_aligned_free|. Otherwise, this class is not
defined. If this class is define, it is the default method if
\verb|MCKL_USE_TBB_MALLOC| is false.

\paragraph{\texttt{AlignedMemorySTD}} This class uses \verb|std::malloc| and
\verb|std::free| and padding the memory manually when the alignment is
insufficient. It is the last resort method that the library will use.

The default alignment depends on the type \verb|T|. If it is a scalar type,
then the alignment is \verb|MCKL_ALIGNMENT|, whose default is 32. This
alignment is sufficient for modern \simd operations, such as \avx. For other
types, the alignment is the maximum of \verb|alignof(T)| and
\verb|MCKL_ALIGNMENT_MIN|, whose default is 16.

Two container types are defined for convenience. The first is a drop-in
replacement for \verb|std::vector|. It is merely a type alias with a different
default allocator,
\begin{Verbatim}
  template <typename T>
  using Vector = std::vector<T, Allocator<T>>;
\end{Verbatim}
Any function template that accepts \verb|std::vector| will also accept this
class, as long as the template is defined correctly. For example,
\begin{Verbatim}
  template <typename T, typename Alloc>
  void func(std::vector<T, Alloc> &vec);
\end{Verbatim}
The second can be used as a drop-in replaced of \verb|std::array| in most
situations,
\begin{Verbatim}
  template <
      typename T,
      size_t N,
      size_t Alignment = AlignmentTrait<T>::value>
  class alignas(Alignment) Array;
\end{Verbatim}
It is a derived class and can be implicitly converted to the standard
library class. The data is in the base class, and thus there will be no loss of
contents. Unlike the standard library versions, they will allocate memory with
bigger alignment for scalar types, on the heap and stack, respectively.

\section{Sample covariance}
\label{sec:Sample covariance}

The library provides the class template
\begin{Verbatim}
  template <typename RealType = double>
  class Covariance;
\end{Verbatim}
to estimate the sample covariance $\Sigma$, for samples $x\in\Real^p$, $p\ge1$,
\begin{align*}
  \Sigma_{i,j} &= \frac{\sum_{i=1}^N w_i}
  {(\sum_{i=1}^N w_i)^2 - \sum_{i=1}^n w_i^2}
  \sum_{k=1}^N w_k (x_{k,i} - \bar{x}_i)(x_{k,j} - \bar{x}_j) \\
  \bar{x}_i &= \frac{1}{\sum_{i=1}^N w_i}\sum_{k=1}^N w_k x_{k,i}
\end{align*}
where $x_{i,j}$ is the $j$\ith component of the $i$\ith sample, and $w_i$ is
the weight of the $i$\ith sample. At the time of writing, only \verb|float| and
\verb|double| are supported. The class has the following operator as its
interface,
\begin{Verbatim}
  void operator()(
      MatrixLayout layout,
      size_t n,
      size_t p,
      const RealType *x,
      const RealType *w,
      RealType *mean,
      RealType *cov,
      MatrixLayout cov_layout = RowMajor,
      bool cov_upper = false,
      bool cov_packed = false)
\end{Verbatim}
It computes the sample covariance matrix $\Sigma$,
where $x$ is the $n$ by $p$ matrix of samples, and $w$ is the $n$-vector of
weights. Below we given detailed description of each parameter,
\begin{description}
  \item[\texttt{layout}] The storage layout of sample matrix $x$.
  \item[\texttt{n}] The number of samples.
  \item[\texttt{p}] The dimension of samples.
  \item[\texttt{x}] The sample matrix. If it is a null pointer, then no
    computation is done.
  \item[\texttt{w}] The weight vector. If it is a null pointer, then $w_i = 1$
    for $i = 1,\dots,n$.
  \item[\texttt{mean}] Output storage of the mean. If it is a null pointer,
    then it is ignored.
  \item[\texttt{cov}] Output storage of the covariance matrix. If it is a null
    pointer, then it is ignored.
  \item[\texttt{cov\_layout}] The storage layout of the covariance matrix.
  \item[\texttt{cov\_upper}] If true, then the upper triangular of the
    covariance matrix is packed, otherwise the lower triangular is packed.
    It is ignored if \verb|cov_pack| is false.
  \item[\texttt{cov\_packed}] If true, then the covariance matrix is
    packed.
\end{description}
The last three parameters specify the storage scheme of the covariance matrix.
See any reference of \blas or \lapack for explanation of the scheme. Below is
an example of the class in use,
\begin{Verbatim}
  using T = StateMatrix<RowMajor, p, double>;
  Sampler<T> sampler(n);
  // Configure and iterate the sampler
  double mean[p];
  double cov[p * (p + 1) / 2];
  Covariance eval;
  auto x = sampler.particle().state().data();
  auto w = sampler.particle().weight().data();
  eval(RowMajor, n, p, x, w, mean, cov, RowMajor, false, true);
\end{Verbatim}
One can later compute the Cholesky decomposition using \lapack or other linear
algebra libraries. Below is an example of using the covariance matrix to
generate multivariate Normal proposals,
\begin{Verbatim}
  double chol[p * (p + 1) / 2];
  double y[p];
  LAPACKE_dpptrf(LAPACK_ROW_MAJOR, 'L', p, chol);
  NormalMVDistribution<double, p> normal_mv(mean, chol);
  normal_mv(rng, y);
\end{Verbatim}

\section{Store objects in \texorpdfstring{\protect\hdf5}{HDF5} format}
\label{sec:Store objects in HDF5 format}

If the \hdf library is available (\verb|MCKL_HAS_HDF5|), it is possible to
store \verb|Sampler| objects, etc., in the \hdf format. For example,
\begin{Verbatim}
  hdf5store(sampler, "pf.h5", "sampler", false);
\end{Verbatim}
creates an \hdf file named \verb|pf.h5| with the sampler stored as a list in
the group \verb|sampler|. If the last argument is \verb|true|, the data is
inserted to an existing file. Otherwise a new file is created. In R it can be
processed as the following,
\begin{Verbatim}
  library(rhdf5)
  pf <- as.data.frame(h5read("pf.h5", "sampler"))
\end{Verbatim}
This creates a \verb|data.frame| similar to that shown in
section~\ref{sub:Implementation (PF)}. The \verb|hdf5store| function is
overloaded for \verb|StateMatrix|, \verb|Sampler| and \verb|Monitor|. It is
also overloaded for \verb|Particle| if an overload for \verb|T| is defined.
They all follow the format as above. In addition, the following creates a new
empty \hdf file,
\begin{Verbatim}
  hdf5store(filename);
\end{Verbatim}
and the following creates a new group named \verb|dataname|,
\begin{Verbatim}
  hdf5store(filename, dataname, append);
\end{Verbatim}

\section{Stop watch}
\label{sec:Stop watch}

Performance can only be improved after it is first properly profiled. There are
advanced profiling programs for this purpose. However, sometime simple timing
facilities are enough. The library provides a simple class \verb|StopWatch| for
this purpose. As its name suggests, it works much like a physical stop watch.
Here is a simple example
\begin{Verbatim}
  StopWatch watch;
  for (int i = 0; i != n; ++i) {
      // Some computation
      watch.start();
      // Computation to be benchmarked;
      watch.stop();
      // Some other computation
  }
  double t = watch.seconds(); // The time in seconds
\end{Verbatim}
The above example demonstrate that timing can be accumulated between loop
iterations, function calls, etc. It shall be noted that, the timing is only
using if the computation between \verb|start| and \verb|stop| is non-trivial.
