% ============================================================================
%  MCKL/manual/tex/config.tex
% ----------------------------------------------------------------------------
%  MCKL: Monte Carlo Kernel Library
% ----------------------------------------------------------------------------
%  Copyright (c) 2013-2016, Yan Zhou
%  All rights reserved.
%
%  Redistribution and use in source and binary forms, with or without
%  modification, are permitted provided that the following conditions are met:
%
%    Redistributions of source code must retain the above copyright notice,
%    this list of conditions and the following disclaimer.
%
%    Redistributions in binary form must reproduce the above copyright notice,
%    this list of conditions and the following disclaimer in the documentation
%    and/or other materials provided with the distribution.
%
%  THIS SOFTWARE IS PROVIDED BY THE COPYRIGHT HOLDERS AND CONTRIBUTORS "AS IS"
%  AND ANY EXPRESS OR IMPLIED WARRANTIES, INCLUDING, BUT NOT LIMITED TO, THE
%  IMPLIED WARRANTIES OF MERCHANTABILITY AND FITNESS FOR A PARTICULAR PURPOSE
%  ARE DISCLAIMED. IN NO EVENT SHALL THE COPYRIGHT HOLDER OR CONTRIBUTORS BE
%  LIABLE FOR ANY DIRECT, INDIRECT, INCIDENTAL, SPECIAL, EXEMPLARY, OR
%  CONSEQUENTIAL DAMAGES (INCLUDING, BUT NOT LIMITED TO, PROCUREMENT OF
%  SUBSTITUTE GOODS OR SERVICES; LOSS OF USE, DATA, OR PROFITS; OR BUSINESS
%  INTERRUPTION) HOWEVER CAUSED AND ON ANY THEORY OF LIABILITY, WHETHER IN
%  CONTRACT, STRICT LIABILITY, OR TORT (INCLUDING NEGLIGENCE OR OTHERWISE)
%  ARISING IN ANY WAY OUT OF THE USE OF THIS SOFTWARE, EVEN IF ADVISED OF THE
%  POSSIBILITY OF SUCH DAMAGE.
% ============================================================================

\chapter{Configuration macros}
\label{chap:Configuration macros}

The library has a few configuration macros. All macros have default values if
left undefined, and can be overwritten by defining them with proper values
before including any of the library headers.

There are three types of configuration macros. The first type has a prefix
\verb|MCKL_HAS_|. These macros specify that a certain feature or third-party
library is available. The second type has a prefix \verb|MCKL_USE_|. These
macros specify that a certain feature or third-party library can be used, if
available. These two kinds of macros shall only be defined to integral values,
with zero as false and all other values as true. The third type affects
implementation choices made by the library. Their possible values vary with
each specific macro.

A concrete example of the relations among these three types of configuration
macros is the memory allocation functions used by the library. The details is
in Section~\ref{sec:Aligned memory allocation}. In short, they library has
three implementations for aligned memory allocation. The preferred method is to
use functions from the \tbb library. It is defined through a class called
\verb|AlignedMemoryTBB|. This class is only defined if and only if
\verb|MCKL_HAS_TBB| is true, in which case the runtime library \verb|tbbmalloc|
will need to be linked as well (\verb|-ltbbmalloc| on \unix-alike systems).
However, if it is desirable to define this class such that the user can use it
with \stl containers, but it is undesirable for the library itself to use it
everywhere, then one can further define the macro \verb|MCKL_USE_TBB_MALLOC| to
zero. In this case, the class is still defined, but it will not be used by the
library. Instead the library will now prefer operating system dependent memory
allocation functions such as \verb|posix_memalign|. Now, suppose this is still
undesirable, and one wants to use a memory allocation function not directly
supported by the library. In this case, one can implement a aligned memory
class (see Section~\ref{sec:Aligned memory allocation} for details), and then
define the macro \verb|MCKL_ALIGNED_MEOMRY_TYPE| to this class. For example, to
use the memory allocation functions from the \mkl library,
\begin{Verbatim}
  #include <mkl.h>

  class AlignedMemoryMKL
  {
      public:
      static void *aligned_malloc(size_t n, size_t alignment)
      {
          return mkl_malloc(n > 0 ? n : 1, alignment);
      }

      static void aligned_free(void *ptr)
      {
          mkl_free(ptr);
      }
  };

  // Use a fully qualified name in macro
  // This will avoid any possible name conflicts
  #define MCKL_ALIGNED_MEMORY_TYPE ::AlignedMemoryMKL

  // Include MCKL headers
\end{Verbatim}
In summary, the \verb|MCKL_HAS_| macros affect what will be \emph{defined} by
the library. The \verb|MCKL_USE_| macros affect what will be \emph{used} by the
library. And other macros give direct control of the library's implementations.
In most situations, it is sufficient to use the first two to configure the
features of the library. The third kind is for more advanced users. Below we
details all configuration macros.

\section{Platform characteristics}
\label{sec:Platform characteristics}

The following macros have default values that depend on the compiler, operating
system or \cpu types. Their default values are therefore platform dependent.
The library tries its best to determine the correct values. But in the case of
errors, define these macros to override the default values.

\paragraph{\texttt{MCKL\_HAS\_X86}} Determine if the \cpu is x86 compatible. In
few rare occasions the library will use features known to exist on x86 \cpu{}s,
for example the little-endian memory layout, etc.

\paragraph{\texttt{MCKL\_HAS\_X86\_64}} Determine if the \cpu is x86-64
compatible.

\paragraph{\texttt{MCKL\_HAS\_POSIX}} Determine if the operating system is
\posix compatible.

\paragraph{\texttt{MCKL\_HAS\_AES\_NI}} Determine if the \aesni instructions
are supported by the \cpu and the compiler. The \rng{}s based on these
instructions in Section~\ref{sub:AES-NI instructions based RNG} are only
defined if this macro is true. One can include the following header directly to
bypass this check,
\begin{Verbatim}
  #include <mckl/random/aesni.hpp>
\end{Verbatim}

\paragraph{\texttt{MCKL\_HAS\_RDRAND}} Determine if the \rdrand instructions
are supported by the \cpu and the compiler. The \rng{}s based on these
instructions in Section~\ref{sec:Non-deterministic RNG} are only defined if
this macro is true. One can include the following header directly to bypass
this check,
\begin{Verbatim}
  #include <mckl/random/rdrand.hpp>
\end{Verbatim}

\paragraph{\texttt{MCKL\_HAS\_INT128}} Determine if the compiler supports
128-bit integer types.

\paragraph{\texttt{MCKL\_INT\_128}} The type of the 128-bit integer. This macro
only has effects if \verb|MCKL_HAS_INT128| is true. The default is
\verb|__int128|.

\section{Third-party libraries}
\label{sec:Third-party libraries}

The following macros determine if certain third-party libraries are available.
All macros with the prefix \verb|MCKL_HAS_| are define to 0 by default. All
macros with the prefix \verb|MCKL_USE_| are defined to 1 if the corresponding
\verb|MCKL_HAS_| macro is true. Otherwise it is defined to 0, unless stated
otherwise.

\paragraph{\texttt{MCKL\_HAS\_OMP}} Determine if OpenMP is supported.

\paragraph{\texttt{MCKL\_USE\_OMP}} Allow the library to use OpenMP for
parallelization. Currently this macro has no affect. The use of OpenMP for
parallelization is explicitly through \verb|SamplerEvalOMP| etc., see
Section~\ref{sec:Programming models}.

\paragraph{\texttt{MCKL\_HAS\_OPENCL}} Determine if OpenCL (version~1.2 or
higher) is supported.

\paragraph{\texttt{MCKL\_HAS\_HDF5}} Determine if the \hdf library is
available. The functions in Section~\ref{sec:Store objects in HDF5 format} are
only defined if this macro is true. One can include the following header
directly to bypass this check,
\begin{Verbatim}
  #include <mckl/utility/hdf5.hpp>
\end{Verbatim}

\paragraph{\texttt{MCKL\_HAS\_TBB}} Determine if the \tbb library is available.
The class template \verb|RNGSetTBB| in Section~\ref{sec:Multiple RNG streams}
and the class \verb|AlignedMemoryTBB| in Section~\ref{sec:Aligned memory
  allocation}. are only defined if this macro is true.

\paragraph{\texttt{MCKL\_USE\_TBB}} Allow the library to use the \tbb library
for parallelization. Currently this macro has no affect. The use of \tbb for
parallelization is explicitly through \verb|SamplerEvalTBB| etc., see
Section~\ref{sec:Programming models}.

\paragraph{\texttt{MCKL\_USE\_TBB\_MALLOC}} Allow the library to use
\verb|AlignedMemoryTBB| as the default memory allocation method.

\paragraph{\texttt{MCKL\_USE\_TBB\_TLS}} Allow the library to use
\verb|RNGSetTBB| as the default multiple \rng stream management method.

\paragraph{\texttt{MCKL\_HAS\_MKL}} Determine if the \mkl library is available.
The \rng{}s in Section~\ref{sec:MKL RNG} are only defined if this macro is
true. One can include the following header directly to bypass this check,
\begin{Verbatim}
  #include <mckl/random/mkl.hpp>
\end{Verbatim}

\paragraph{\texttt{MCKL\_USE\_MKL\_CBLAS}} Allow the library to use the
\verb|mkl_cblas.h| header and assuming that corresponding runtime library will
be linked for \blas support.

\paragraph{\texttt{MCKL\_USE\_MKL\_VML}} Allow the library to accelerate the
vector functions in Section~\ref{sec:Vectorized functions} using the \vml
component of \mkl.

\paragraph{\texttt{MCKL\_USE\_MKL\_VSL}} Allow the library to use the \vsl
component of \mkl.

\paragraph{\texttt{MCKL\_USE\_CBLAS}} Allow the library to the standard C
interface of \blas. The default is \verb|MCKL_USE_MKL_CBLAS|. Manually define
this macro to true if the header file \verb|cblas.h| is available and a
compatible runtime library is linked. When this macro is tested to be false,
the library will declare the \blas Fortran routines in C itself (see below).

\paragraph{\texttt{MCKL\_BLAS\_NAME} and
  \texttt{MCKL\_BLAS\_NAME\_NO\_UNDERSCORE}} These two macros determine how
shall the \blas Fortran routines be declared in C. The default behavior is to
append an underscore to the function name. For example, \verb|dgemv| in Fortran
becomes \verb|dgemv_| in C. If there should be no underscore, define the second
macro. If the name mangling is more complicated, one can define the
\verb|MCKL_BLAS_NAME| macro directly. For example,
\begin{Verbatim}
  #define MCKL_BLAS_NAME(x) _##x
\end{Verbatim}
These macros only have effects if \verb|MCKL_USE_CBLAS| is false.

\paragraph{\texttt{MCKL\_BLAS\_INT}} The integer type of \blas routines. The
default is \verb|MKL_INT| if \verb|MCKL_USE_CBLAS| and
\verb|MCKL_USE_MKL_CBLAS| are both true. Otherwise it is \verb|int|. If the
\blas interface uses another integer type, such as the \ilp{}64 interface of
some implementations on \lp{}64 platforms, then one should redefine this macro
to the correct type.

\section{\texorpdfstring{\protect\rng}{RNG} engines}
\label{sec:RNG engines}

Configuration macros related to \rng{}s engines are discussed in
Section~\ref{sec:Counter-based RNG} and~\ref{sec:Multiple RNG streams}.

\section{Memory allocation}
\label{sec:Memory allocation}

\paragraph{\texttt{MCKL\_ALIGNMENT}} The default alignment for scalar types,
such as \verb|int|. More specifically, this affects types such that
\verb|std::is_scalar<T>| true. The default is~32. This is sufficient for modern
\simd operations. This will affect the memory allocated by \verb|Vector<T>| on
the heap and \verb|Array<T>| on the stack. The value must be a power of two and
positive.

\paragraph{\texttt{MCKL\_ALIGNMENT\_MIN}} The minimum alignment for all types.
The default is 16. The value must be a power of two and positive.

\paragraph{\texttt{MCKL\_ALIGNED\_MEMORY\_TYPE}} The default type of the
\verb|Memroy| template parameter of \verb|Allocator| (see
Section~\ref{sec:Aligned memory allocation}). The default depends on the other
configuration macros. If \verb|MCKL_USE_TBB_MALLOC| is true, then the
default is \verb|AlignedMemoryTBB|. Otherwise it is \verb|AlignedMemorySYS| if
it is defined. Otherwise \verb|AlignedMemmorySTD| is used as a last resort.

\paragraph{\texttt{MCKL\_CONSTRUCT\_SCALAR}} Determine if scalar types shall be
zero initialized upon allocation by \verb|Allocator|. The default is 0. This
departures from the behavior of \verb|std::allocator|, with which the memory is
always value initialized. For example,
\begin{Verbatim}
  std::vector<int> v(n);
\end{Verbatim}
will initialize all values to zero. In contrast,
\begin{Verbatim}
  Vector<int> v(n);
\end{Verbatim}
will leave the memory uninitialized. To get zero initialized vectors, one can
either define this macro to 1, or more efficiently,
\begin{Verbatim}
  Vector<int> v(n);
  std::fill(v.begin(), v.end(), 0); // or
  std::memset(v.data(), 0, sizeof(int) * v.size());
\end{Verbatim}
The two methods are equivalent in most standard library implementations. It is
strongly recommended that to leave this macro defined to zero. The constructor
of \verb|std::vector|, with which \verb|Vector| is an alias to, calls the
\verb|construct| member of \verb|Allocator| to initialize elements one by one,
which is a huge waste of time. The library does not rely on zero initialization
itself. Note that, this macro only affect \emph{default initialization of
  scalars}. For class types etc., and other constructor of \verb|std::vector|,
such as
\begin{Verbatim}
  Vector<ClassType> c(n);
  Vector<int> v(n, 2);
\end{Verbatim}
the behavior will be as expected. That is, the first will be default
initialized and the second will be value initialized.

\section{Error handling}
\label{sec:Error handling}

\paragraph{\texttt{MCKL\_NO\_RUNTIME\_ASSERT}} Determine if all runtime
assertions shall be disabled. The default is 1 if \verb|NDEBUG| is defined.
Otherwise, it is 0. Runtime assertions are hard errors. The program will be
terminated by \verb|std::exit| upon failure of assertions. These are usually
errors that will cause undefined behaviors.

\paragraph{\texttt{MCKL\_RUNTIME\_ASSERT\_AS\_EXCEPTION}} Determine if runtime
assertions shall be turned into exceptions. The exception type is
\verb|RuntimeAssert|. The default is 0. It has no affect is
\verb|MCKL_NO_RUNTIME_ASSERT| is true.
