% ============================================================================
%  MCKL/manual/tex/perf_random_rng.tex
% ----------------------------------------------------------------------------
%  MCKL: Monte Carlo Kernel Library
% ----------------------------------------------------------------------------
%  Copyright (c) 2013-2016, Yan Zhou
%  All rights reserved.
%
%  Redistribution and use in source and binary forms, with or without
%  modification, are permitted provided that the following conditions are met:
%
%    Redistributions of source code must retain the above copyright notice,
%    this list of conditions and the following disclaimer.
%
%    Redistributions in binary form must reproduce the above copyright notice,
%    this list of conditions and the following disclaimer in the documentation
%    and/or other materials provided with the distribution.
%
%  THIS SOFTWARE IS PROVIDED BY THE COPYRIGHT HOLDERS AND CONTRIBUTORS "AS IS"
%  AND ANY EXPRESS OR IMPLIED WARRANTIES, INCLUDING, BUT NOT LIMITED TO, THE
%  IMPLIED WARRANTIES OF MERCHANTABILITY AND FITNESS FOR A PARTICULAR PURPOSE
%  ARE DISCLAIMED. IN NO EVENT SHALL THE COPYRIGHT HOLDER OR CONTRIBUTORS BE
%  LIABLE FOR ANY DIRECT, INDIRECT, INCIDENTAL, SPECIAL, EXEMPLARY, OR
%  CONSEQUENTIAL DAMAGES (INCLUDING, BUT NOT LIMITED TO, PROCUREMENT OF
%  SUBSTITUTE GOODS OR SERVICES; LOSS OF USE, DATA, OR PROFITS; OR BUSINESS
%  INTERRUPTION) HOWEVER CAUSED AND ON ANY THEORY OF LIABILITY, WHETHER IN
%  CONTRACT, STRICT LIABILITY, OR TORT (INCLUDING NEGLIGENCE OR OTHERWISE)
%  ARISING IN ANY WAY OUT OF THE USE OF THIS SOFTWARE, EVEN IF ADVISED OF THE
%  POSSIBILITY OF SUCH DAMAGE.
% ============================================================================

\chapter{Performance of \texorpdfstring{\protect{rng}}{RNG}}
\label{chap:Performance of RNG}

The performance of \rng{}s in sections~\ref{sec:Counter-based RNG}
to~\ref{sec:Non-deterministic RNG} are measured in single core cycles per byte
(cpB). The processor used is Intel Core i7-4960\textsc{hq}. The operating
system is Mac OS X (version~10.11.4).

Three compilers are tested. The \llvm clang (version Apple~7.3.0), the \gnu{}
\gcc (version~5.3) and the Intel \cpp compiler (version~2016 update~3). They
are labeled ``\llvm'', ``\gcc'', and ``\textsc{intel}'', respectively, in
tables below.

Two usage cases of \rng{}s are considered. The first is generating random
integers within a loop,
\begin{verbatim}
UniformBitsDistribution<uint64_t> ubits;
for (std:size_t i = 0; i != n; ++i)
    r[i] = rand(rng, ubits);
\end{verbatim}
The second is the vectorized performance,
\begin{verbatim}
rand(rng, ubits, n, r.data());
\end{verbatim}
In both cases, we repeat the simulations 100 times, each time with the number
of elements $n$ chosen randomly between 5,000 and 10,000. The total number of
cycles of the 100 simulations are recorded, and then divided by the total
number of bytes generated. This gives the performance measurement in cpB. This
experiment is repeated ten times, and the best results are shown. The two cases
are labeled ``\textsc{loop}` and ``\textsc{batch}'', respectively, in the
tables below.

\begin{table}
  \input{tab/random_rng_std}%
  \caption{Performance of \protect\rng{}s in the standard library}
  \label{tab:Performance of standard library RNG}
\end{table}

\begin{table}
  \input{tab/random_rng_aes128}%
  \caption{Performance of \protect\texttt{AES128Engine}}
  \label{tab:Performance of AES128Engine}
\end{table}

\begin{table}
  \input{tab/random_rng_aes192}%
  \caption{Performance of \protect\texttt{AES192Engine}}
  \label{tab:Performance of AES192Engine}
\end{table}

\begin{table}
  \input{tab/random_rng_aes256}%
  \caption{Performance of \protect\texttt{AES256Engine}}
  \label{tab:Performance of AES256Engine}
\end{table}

\begin{table}
  \input{tab/random_rng_ars}%
  \caption{Performance of \protect\texttt{ARSEngine}}
  \label{tab:Performance of ARSEngine}
\end{table}

\begin{table}
  \input{tab/random_rng_philox}%
  \caption{Performance of \protect\texttt{PhiloxEngine}}
  \label{tab:Performance of PhiloxEngine}
\end{table}

\begin{table}
  \input{tab/random_rng_threefry}%
  \caption{Performance of \protect\texttt{ThreefryEngine}}
  \label{tab:Performance of ThreefryEngine}
\end{table}

\begin{table}
  \input{tab/random_rng_rdrand}%
  \caption{Performance of non-deterministic \protect\rng{}s}
  \label{tab:Performance of non-deterministic RNG}
\end{table}

\begin{table}
  \input{tab/random_rng_mkl}%
  \caption{Performance of \protect\mkl{} \protect\rng{}s}
  \label{tab:Performance of MKL RNG}
\end{table}
